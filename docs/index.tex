% Options for packages loaded elsewhere
% Options for packages loaded elsewhere
\PassOptionsToPackage{unicode}{hyperref}
\PassOptionsToPackage{hyphens}{url}
\PassOptionsToPackage{dvipsnames,svgnames,x11names}{xcolor}
%
\documentclass[
  letterpaper,
  DIV=11,
  numbers=noendperiod]{scrreprt}
\usepackage{xcolor}
\usepackage{amsmath,amssymb}
\setcounter{secnumdepth}{5}
\usepackage{iftex}
\ifPDFTeX
  \usepackage[T1]{fontenc}
  \usepackage[utf8]{inputenc}
  \usepackage{textcomp} % provide euro and other symbols
\else % if luatex or xetex
  \usepackage{unicode-math} % this also loads fontspec
  \defaultfontfeatures{Scale=MatchLowercase}
  \defaultfontfeatures[\rmfamily]{Ligatures=TeX,Scale=1}
\fi
\usepackage{lmodern}
\ifPDFTeX\else
  % xetex/luatex font selection
\fi
% Use upquote if available, for straight quotes in verbatim environments
\IfFileExists{upquote.sty}{\usepackage{upquote}}{}
\IfFileExists{microtype.sty}{% use microtype if available
  \usepackage[]{microtype}
  \UseMicrotypeSet[protrusion]{basicmath} % disable protrusion for tt fonts
}{}
\makeatletter
\@ifundefined{KOMAClassName}{% if non-KOMA class
  \IfFileExists{parskip.sty}{%
    \usepackage{parskip}
  }{% else
    \setlength{\parindent}{0pt}
    \setlength{\parskip}{6pt plus 2pt minus 1pt}}
}{% if KOMA class
  \KOMAoptions{parskip=half}}
\makeatother
% Make \paragraph and \subparagraph free-standing
\makeatletter
\ifx\paragraph\undefined\else
  \let\oldparagraph\paragraph
  \renewcommand{\paragraph}{
    \@ifstar
      \xxxParagraphStar
      \xxxParagraphNoStar
  }
  \newcommand{\xxxParagraphStar}[1]{\oldparagraph*{#1}\mbox{}}
  \newcommand{\xxxParagraphNoStar}[1]{\oldparagraph{#1}\mbox{}}
\fi
\ifx\subparagraph\undefined\else
  \let\oldsubparagraph\subparagraph
  \renewcommand{\subparagraph}{
    \@ifstar
      \xxxSubParagraphStar
      \xxxSubParagraphNoStar
  }
  \newcommand{\xxxSubParagraphStar}[1]{\oldsubparagraph*{#1}\mbox{}}
  \newcommand{\xxxSubParagraphNoStar}[1]{\oldsubparagraph{#1}\mbox{}}
\fi
\makeatother

\usepackage{color}
\usepackage{fancyvrb}
\newcommand{\VerbBar}{|}
\newcommand{\VERB}{\Verb[commandchars=\\\{\}]}
\DefineVerbatimEnvironment{Highlighting}{Verbatim}{commandchars=\\\{\}}
% Add ',fontsize=\small' for more characters per line
\usepackage{framed}
\definecolor{shadecolor}{RGB}{241,243,245}
\newenvironment{Shaded}{\begin{snugshade}}{\end{snugshade}}
\newcommand{\AlertTok}[1]{\textcolor[rgb]{0.68,0.00,0.00}{#1}}
\newcommand{\AnnotationTok}[1]{\textcolor[rgb]{0.37,0.37,0.37}{#1}}
\newcommand{\AttributeTok}[1]{\textcolor[rgb]{0.40,0.45,0.13}{#1}}
\newcommand{\BaseNTok}[1]{\textcolor[rgb]{0.68,0.00,0.00}{#1}}
\newcommand{\BuiltInTok}[1]{\textcolor[rgb]{0.00,0.23,0.31}{#1}}
\newcommand{\CharTok}[1]{\textcolor[rgb]{0.13,0.47,0.30}{#1}}
\newcommand{\CommentTok}[1]{\textcolor[rgb]{0.37,0.37,0.37}{#1}}
\newcommand{\CommentVarTok}[1]{\textcolor[rgb]{0.37,0.37,0.37}{\textit{#1}}}
\newcommand{\ConstantTok}[1]{\textcolor[rgb]{0.56,0.35,0.01}{#1}}
\newcommand{\ControlFlowTok}[1]{\textcolor[rgb]{0.00,0.23,0.31}{\textbf{#1}}}
\newcommand{\DataTypeTok}[1]{\textcolor[rgb]{0.68,0.00,0.00}{#1}}
\newcommand{\DecValTok}[1]{\textcolor[rgb]{0.68,0.00,0.00}{#1}}
\newcommand{\DocumentationTok}[1]{\textcolor[rgb]{0.37,0.37,0.37}{\textit{#1}}}
\newcommand{\ErrorTok}[1]{\textcolor[rgb]{0.68,0.00,0.00}{#1}}
\newcommand{\ExtensionTok}[1]{\textcolor[rgb]{0.00,0.23,0.31}{#1}}
\newcommand{\FloatTok}[1]{\textcolor[rgb]{0.68,0.00,0.00}{#1}}
\newcommand{\FunctionTok}[1]{\textcolor[rgb]{0.28,0.35,0.67}{#1}}
\newcommand{\ImportTok}[1]{\textcolor[rgb]{0.00,0.46,0.62}{#1}}
\newcommand{\InformationTok}[1]{\textcolor[rgb]{0.37,0.37,0.37}{#1}}
\newcommand{\KeywordTok}[1]{\textcolor[rgb]{0.00,0.23,0.31}{\textbf{#1}}}
\newcommand{\NormalTok}[1]{\textcolor[rgb]{0.00,0.23,0.31}{#1}}
\newcommand{\OperatorTok}[1]{\textcolor[rgb]{0.37,0.37,0.37}{#1}}
\newcommand{\OtherTok}[1]{\textcolor[rgb]{0.00,0.23,0.31}{#1}}
\newcommand{\PreprocessorTok}[1]{\textcolor[rgb]{0.68,0.00,0.00}{#1}}
\newcommand{\RegionMarkerTok}[1]{\textcolor[rgb]{0.00,0.23,0.31}{#1}}
\newcommand{\SpecialCharTok}[1]{\textcolor[rgb]{0.37,0.37,0.37}{#1}}
\newcommand{\SpecialStringTok}[1]{\textcolor[rgb]{0.13,0.47,0.30}{#1}}
\newcommand{\StringTok}[1]{\textcolor[rgb]{0.13,0.47,0.30}{#1}}
\newcommand{\VariableTok}[1]{\textcolor[rgb]{0.07,0.07,0.07}{#1}}
\newcommand{\VerbatimStringTok}[1]{\textcolor[rgb]{0.13,0.47,0.30}{#1}}
\newcommand{\WarningTok}[1]{\textcolor[rgb]{0.37,0.37,0.37}{\textit{#1}}}

\usepackage{longtable,booktabs,array}
\usepackage{calc} % for calculating minipage widths
% Correct order of tables after \paragraph or \subparagraph
\usepackage{etoolbox}
\makeatletter
\patchcmd\longtable{\par}{\if@noskipsec\mbox{}\fi\par}{}{}
\makeatother
% Allow footnotes in longtable head/foot
\IfFileExists{footnotehyper.sty}{\usepackage{footnotehyper}}{\usepackage{footnote}}
\makesavenoteenv{longtable}
\usepackage{graphicx}
\makeatletter
\newsavebox\pandoc@box
\newcommand*\pandocbounded[1]{% scales image to fit in text height/width
  \sbox\pandoc@box{#1}%
  \Gscale@div\@tempa{\textheight}{\dimexpr\ht\pandoc@box+\dp\pandoc@box\relax}%
  \Gscale@div\@tempb{\linewidth}{\wd\pandoc@box}%
  \ifdim\@tempb\p@<\@tempa\p@\let\@tempa\@tempb\fi% select the smaller of both
  \ifdim\@tempa\p@<\p@\scalebox{\@tempa}{\usebox\pandoc@box}%
  \else\usebox{\pandoc@box}%
  \fi%
}
% Set default figure placement to htbp
\def\fps@figure{htbp}
\makeatother





\setlength{\emergencystretch}{3em} % prevent overfull lines

\providecommand{\tightlist}{%
  \setlength{\itemsep}{0pt}\setlength{\parskip}{0pt}}



 


\KOMAoption{captions}{tableheading}
\makeatletter
\@ifpackageloaded{bookmark}{}{\usepackage{bookmark}}
\makeatother
\makeatletter
\@ifpackageloaded{caption}{}{\usepackage{caption}}
\AtBeginDocument{%
\ifdefined\contentsname
  \renewcommand*\contentsname{Table of contents}
\else
  \newcommand\contentsname{Table of contents}
\fi
\ifdefined\listfigurename
  \renewcommand*\listfigurename{List of Figures}
\else
  \newcommand\listfigurename{List of Figures}
\fi
\ifdefined\listtablename
  \renewcommand*\listtablename{List of Tables}
\else
  \newcommand\listtablename{List of Tables}
\fi
\ifdefined\figurename
  \renewcommand*\figurename{Figure}
\else
  \newcommand\figurename{Figure}
\fi
\ifdefined\tablename
  \renewcommand*\tablename{Table}
\else
  \newcommand\tablename{Table}
\fi
}
\@ifpackageloaded{float}{}{\usepackage{float}}
\floatstyle{ruled}
\@ifundefined{c@chapter}{\newfloat{codelisting}{h}{lop}}{\newfloat{codelisting}{h}{lop}[chapter]}
\floatname{codelisting}{Listing}
\newcommand*\listoflistings{\listof{codelisting}{List of Listings}}
\makeatother
\makeatletter
\makeatother
\makeatletter
\@ifpackageloaded{caption}{}{\usepackage{caption}}
\@ifpackageloaded{subcaption}{}{\usepackage{subcaption}}
\makeatother
\usepackage{bookmark}
\IfFileExists{xurl.sty}{\usepackage{xurl}}{} % add URL line breaks if available
\urlstyle{same}
\hypersetup{
  pdftitle={Documentación Técnica de UpTrackAI},
  pdfauthor={Equipo UpTrackAI},
  colorlinks=true,
  linkcolor={blue},
  filecolor={Maroon},
  citecolor={Blue},
  urlcolor={Blue},
  pdfcreator={LaTeX via pandoc}}


\title{Documentación Técnica de UpTrackAI}
\author{Equipo UpTrackAI}
\date{2026-01-05}
\begin{document}
\maketitle

\renewcommand*\contentsname{Table of contents}
{
\hypersetup{linkcolor=}
\setcounter{tocdepth}{2}
\tableofcontents
}

\bookmarksetup{startatroot}

\chapter{Documentación Técnica de
UpTrackAI}\label{documentaciuxf3n-tuxe9cnica-de-uptrackai}

Bienvenido a la documentación técnica de UpTrackAI.

\section{Resumen}\label{resumen}

UpTrackAI es un sistema de monitoreo completo construido con Go y React,
con detección inteligente de estados, notificaciones asíncronas y
procesamiento concurrente.

\section{Arquitectura}\label{arquitectura}

Ver la \href{arquitectura/}{sección de arquitectura} para el diseño
detallado del sistema.

\section{Primeros Pasos}\label{primeros-pasos}

\href{primeros-pasos.qmd}{Instrucciones de instalación y configuración}

\part{Arquitectura del Sistema}

Esta sección describe la arquitectura de UpTrackAI.

\section*{Resumen}\label{resumen-1}
\addcontentsline{toc}{section}{Resumen}

\markright{Resumen}

UpTrackAI sigue los principios de Clean Architecture con las siguientes
capas:

\begin{itemize}
\tightlist
\item
  \textbf{Dominio}: Lógica de negocio central y entidades
\item
  \textbf{Aplicación}: Casos de uso y servicios de aplicación
\item
  \textbf{Infraestructura}: Preocupaciones externas (base de datos,
  APIs, etc.)
\item
  \textbf{Presentación}: Manejadores HTTP y respuestas de API
\end{itemize}

\section*{Componentes}\label{componentes}
\addcontentsline{toc}{section}{Componentes}

\markright{Componentes}

\subsection*{Backend (Go)}\label{backend-go}
\addcontentsline{toc}{subsection}{Backend (Go)}

\begin{itemize}
\tightlist
\item
  \textbf{Módulo de Monitoreo}: Maneja objetivos de monitoreo y
  verificaciones de salud
\item
  \textbf{Módulo de Notificaciones}: Gestiona notificaciones asíncronas
  vía Telegram
\item
  \textbf{Módulo de Usuario}: Gestión de usuarios y autenticación
\item
  \textbf{Módulo de Seguridad}: Seguridad y autorización
\end{itemize}

\subsection*{Frontend
(React/TypeScript)}\label{frontend-reacttypescript}
\addcontentsline{toc}{subsection}{Frontend (React/TypeScript)}

\begin{itemize}
\tightlist
\item
  \textbf{Dashboard}: Interfaz principal de monitoreo
\item
  \textbf{Gestión de Objetivos}: Agregar y configurar objetivos de
  monitoreo
\item
  \textbf{Perfil de Usuario}: Configuraciones de usuario y vinculación
  con Telegram
\end{itemize}

\subsection*{Base de Datos (PostgreSQL)}\label{base-de-datos-postgresql}
\addcontentsline{toc}{subsection}{Base de Datos (PostgreSQL)}

\begin{itemize}
\tightlist
\item
  \textbf{Datos de Monitoreo}: Objetivos, resultados de verificaciones,
  métricas
\item
  \textbf{Datos de Usuario}: Usuarios, canales de notificación
\item
  \textbf{Datos del Sistema}: Configuraciones y migraciones
\end{itemize}

\chapter{Resumen de Arquitectura}\label{resumen-de-arquitectura}

\section{Clean Architecture}\label{clean-architecture}

UpTrackAI implementa Clean Architecture con clara separación de
responsabilidades:

\begin{verbatim}
Capa de Presentación (HTTP, CLI)
    ↓
Capa de Aplicación (Casos de Uso, Servicios)
    ↓
Capa de Dominio (Entidades, Reglas de Negocio)
    ↓
Capa de Infraestructura (Base de Datos, APIs Externas)
\end{verbatim}

\section{Patrones de Diseño Clave}\label{patrones-de-diseuxf1o-clave}

\subsection{Patrón Repository}\label{patruxf3n-repository}

\begin{itemize}
\tightlist
\item
  Abstrae operaciones de acceso a datos
\item
  Permite pruebas y mocking fáciles
\item
  Soporta múltiples implementaciones de base de datos
\end{itemize}

\subsection{Inyección de
Dependencias}\label{inyecciuxf3n-de-dependencias}

\begin{itemize}
\tightlist
\item
  Acoplamiento bajo entre componentes
\item
  Pruebas y mantenimiento más fáciles
\item
  Configurable en tiempo de ejecución
\end{itemize}

\subsection{Patrón Worker Pool}\label{patruxf3n-worker-pool}

\begin{itemize}
\tightlist
\item
  Procesamiento concurrente de objetivos de monitoreo
\item
  Cantidad de workers configurable según carga del sistema
\item
  Apagado graceful y gestión de recursos
\end{itemize}

\section{Modelo de Concurrencia}\label{modelo-de-concurrencia}

El sistema utiliza goroutines y canales para concurrencia:

\begin{itemize}
\tightlist
\item
  \textbf{Worker Pool}: Procesamiento paralelo de objetivos
\item
  \textbf{Notificaciones Asíncronas}: Entrega desacoplada de alertas
\item
  \textbf{Polling de Telegram}: Procesamiento de mensajes en segundo
  plano
\end{itemize}

\section{Gestión de Estados}\label{gestiuxf3n-de-estados}

Monitoreo inteligente con detección de 6 estados:

\begin{itemize}
\tightlist
\item
  \textbf{UP}: El objetivo está saludable
\item
  \textbf{DOWN}: El objetivo es inalcanzable
\item
  \textbf{DEGRADED}: Problemas de rendimiento detectados
\item
  \textbf{UNSTABLE}: Fallos intermitentes
\item
  \textbf{FLAPPING}: Cambios rápidos de estado
\item
  \textbf{UNKNOWN}: Datos insuficientes para determinar
\end{itemize}

\chapter{Arquitectura del Backend}\label{arquitectura-del-backend}

\section{Stack Tecnológico}\label{stack-tecnoluxf3gico}

\begin{itemize}
\tightlist
\item
  \textbf{Lenguaje}: Go 1.21+
\item
  \textbf{Framework}: Gin (enrutador HTTP)
\item
  \textbf{Base de Datos}: PostgreSQL con GORM
\item
  \textbf{Migraciones}: GORM AutoMigrate
\item
  \textbf{Configuración}: Variables de entorno
\item
  \textbf{Logging}: Logging estructurado con contexto
\end{itemize}

\section{Estructura de Módulos}\label{estructura-de-muxf3dulos}

\begin{verbatim}
internal/
├── monitoring/          # Objetivos de monitoreo y verificaciones
├── notifications/       # Sistema de notificaciones asíncronas
├── user/               # Gestión de usuarios
├── security/           # Autenticación y autorización
├── server/             # Servidor HTTP y middleware
└── repository/         # Interfaces de repositorio compartidas
\end{verbatim}

\section{Componentes Clave}\label{componentes-clave}

\subsection{Scheduler de Monitoreo}\label{scheduler-de-monitoreo}

\begin{itemize}
\tightlist
\item
  \textbf{Orchestrator}: Coordina el flujo de trabajo de monitoreo
\item
  \textbf{Worker Pool}: Procesamiento concurrente de objetivos
\item
  \textbf{Health Checker}: Realiza verificaciones reales
\item
  \textbf{Metrics Calculator}: Calcula métricas de rendimiento
\item
  \textbf{Result Analyzer}: Determina el estado del objetivo
\item
  \textbf{Notification Dispatcher}: Dispara alertas
\end{itemize}

\subsection{Sistema de Notificaciones}\label{sistema-de-notificaciones}

\begin{itemize}
\tightlist
\item
  \textbf{Cola Asíncrona}: Canal bufferizado para desacoplamiento
\item
  \textbf{Integración con Telegram}: API de bot con magic links
\item
  \textbf{Persistencia}: Almacenamiento de canales basado en GORM
\item
  \textbf{Servicio de Polling}: Procesamiento de mensajes en segundo
  plano
\end{itemize}

\subsection{Capa de Base de Datos}\label{capa-de-base-de-datos}

\begin{itemize}
\tightlist
\item
  \textbf{Repositorios}: Abstracción de acceso a datos
\item
  \textbf{Migraciones}: Gestión de esquemas
\item
  \textbf{Transacciones}: Cumplimiento ACID para operaciones críticas
\item
  \textbf{Pooling de Conexiones}: Uso eficiente de recursos
\end{itemize}

\section{Diseño de API}\label{diseuxf1o-de-api}

API RESTful con respuestas JSON:

\begin{itemize}
\tightlist
\item
  \texttt{GET\ /api/targets} - Listar objetivos de monitoreo
\item
  \texttt{POST\ /api/targets} - Crear nuevo objetivo
\item
  \texttt{GET\ /api/targets/\{id\}} - Obtener detalles del objetivo
\item
  \texttt{PUT\ /api/targets/\{id\}} - Actualizar objetivo
\item
  \texttt{DELETE\ /api/targets/\{id\}} - Eliminar objetivo
\end{itemize}

\section{Manejo de Errores}\label{manejo-de-errores}

\begin{itemize}
\tightlist
\item
  \textbf{Errores de Dominio}: Validación de lógica de negocio
\item
  \textbf{Errores de Infraestructura}: Fallos de servicios externos
\item
  \textbf{Errores HTTP}: Códigos de estado y mensajes apropiados
\item
  \textbf{Logging}: Reporte estructurado de errores
\end{itemize}

\chapter{Arquitectura del Frontend}\label{arquitectura-del-frontend}

\section{Stack Tecnológico}\label{stack-tecnoluxf3gico-1}

\begin{itemize}
\tightlist
\item
  \textbf{Lenguaje}: TypeScript
\item
  \textbf{Framework}: React 18 con Vite
\item
  \textbf{Estilos}: Tailwind CSS
\item
  \textbf{Gestión de Estado}: Hooks de React + Context
\item
  \textbf{Enrutamiento}: React Router
\item
  \textbf{Cliente API}: Fetch API con hooks personalizados
\item
  \textbf{Herramienta de Build}: Vite
\end{itemize}

\section{Estructura de Componentes}\label{estructura-de-componentes}

\begin{verbatim}
src/
├── components/
│   ├── layout/          # Componentes de layout (Header, Sidebar)
│   └── ui/             # Componentes UI reutilizables
├── pages/              # Componentes de página
│   ├── Dashboard.tsx
│   ├── Systems.tsx
│   ├── AddTarget.tsx
│   └── TargetDetail.tsx
├── api/                # Funciones cliente API
├── data/               # Tipos de datos y constantes
└── assets/             # Activos estáticos
\end{verbatim}

\section{Características Clave}\label{caracteruxedsticas-clave}

\subsection{Dashboard}\label{dashboard}

\begin{itemize}
\tightlist
\item
  Estado de monitoreo en tiempo real
\item
  Resumen de salud de objetivos
\item
  Alertas y notificaciones recientes
\item
  Gráficos de métricas de rendimiento
\end{itemize}

\subsection{Gestión de Objetivos}\label{gestiuxf3n-de-objetivos}

\begin{itemize}
\tightlist
\item
  Agregar/Editar objetivos de monitoreo
\item
  Configurar intervalos de verificación
\item
  Establecer umbrales de alerta
\item
  Ver información detallada del objetivo
\end{itemize}

\subsection{Integración de Usuario}\label{integraciuxf3n-de-usuario}

\begin{itemize}
\tightlist
\item
  Gestión de perfil
\item
  Vinculación con bot de Telegram vía magic links
\item
  Preferencias de notificación
\end{itemize}

\section{Gestión de Estado}\label{gestiuxf3n-de-estado}

Utiliza gestión de estado integrada de React:

\begin{itemize}
\tightlist
\item
  \textbf{Estado Local}: useState para estado de componentes
\item
  \textbf{Estado del Servidor}: Hooks personalizados para datos API
\item
  \textbf{Estado Global}: Context para sesión de usuario
\item
  \textbf{Estado de Formularios}: Componentes controlados
\end{itemize}

\section{Integración con API}\label{integraciuxf3n-con-api}

Hooks personalizados para obtención de datos:

\begin{Shaded}
\begin{Highlighting}[]
\KeywordTok{const}\NormalTok{ useTargets }\OperatorTok{=}\NormalTok{ () }\KeywordTok{=\textgreater{}}\NormalTok{ \{}
  \KeywordTok{const}\NormalTok{ [targets}\OperatorTok{,}\NormalTok{ setTargets] }\OperatorTok{=} \FunctionTok{useState}\NormalTok{([])}\OperatorTok{;}
  \KeywordTok{const}\NormalTok{ [loading}\OperatorTok{,}\NormalTok{ setLoading] }\OperatorTok{=} \FunctionTok{useState}\NormalTok{(}\KeywordTok{true}\NormalTok{)}\OperatorTok{;}

  \FunctionTok{useEffect}\NormalTok{(() }\KeywordTok{=\textgreater{}}\NormalTok{ \{}
    \FunctionTok{fetchTargets}\NormalTok{()}\OperatorTok{.}\FunctionTok{then}\NormalTok{(setTargets)}\OperatorTok{.}\FunctionTok{finally}\NormalTok{(() }\KeywordTok{=\textgreater{}} \FunctionTok{setLoading}\NormalTok{(}\KeywordTok{false}\NormalTok{))}\OperatorTok{;}
\NormalTok{  \}}\OperatorTok{,}\NormalTok{ [])}\OperatorTok{;}

  \ControlFlowTok{return}\NormalTok{ \{ targets}\OperatorTok{,}\NormalTok{ loading \}}\OperatorTok{;}
\NormalTok{\}}\OperatorTok{;}
\end{Highlighting}
\end{Shaded}

\section{Diseño Responsivo}\label{diseuxf1o-responsivo}

Enfoque mobile-first con Tailwind CSS:

\begin{itemize}
\tightlist
\item
  \textbf{Sistema de Breakpoints}: breakpoints sm/md/lg/xl
\item
  \textbf{Layout Grid}: Flexbox y CSS Grid
\item
  \textbf{Biblioteca de Componentes}: Componentes UI consistentes
\item
  \textbf{Modo Oscuro}: Capacidad de cambio de tema
\end{itemize}

\chapter{Arquitectura de Base de
Datos}\label{arquitectura-de-base-de-datos}

\section{Diseño de Base de Datos}\label{diseuxf1o-de-base-de-datos}

Base de datos relacional PostgreSQL con las siguientes tablas clave:

\subsection{Tablas de Monitoreo Core}\label{tablas-de-monitoreo-core}

\begin{itemize}
\tightlist
\item
  \textbf{monitoring\_targets}: Configuraciones de objetivos

  \begin{itemize}
  \tightlist
  \item
    id (UUID, PK)
  \item
    name (varchar)
  \item
    url (varchar)
  \item
    check\_interval (int)
  \item
    timeout (int)
  \item
    expected\_status (int)
  \item
    created\_at, updated\_at (timestamp)
  \end{itemize}
\item
  \textbf{check\_results}: Resultados individuales de verificaciones

  \begin{itemize}
  \tightlist
  \item
    id (UUID, PK)
  \item
    target\_id (UUID, FK)
  \item
    status (enum: UP, DOWN, etc.)
  \item
    response\_time (int)
  \item
    status\_code (int)
  \item
    error\_message (text)
  \item
    checked\_at (timestamp)
  \end{itemize}
\item
  \textbf{target\_metrics}: Datos de rendimiento agregados

  \begin{itemize}
  \tightlist
  \item
    id (UUID, PK)
  \item
    target\_id (UUID, FK)
  \item
    uptime\_percentage (decimal)
  \item
    avg\_response\_time (int)
  \item
    total\_checks (int)
  \item
    period\_start, period\_end (timestamp)
  \end{itemize}
\end{itemize}

\subsection{Tablas de Notificaciones}\label{tablas-de-notificaciones}

\begin{itemize}
\tightlist
\item
  \textbf{notification\_channels}: Preferencias de notificación de
  usuario

  \begin{itemize}
  \tightlist
  \item
    id (varchar(100), PK) - ID de canal de Telegram
  \item
    user\_id (UUID, FK)
  \item
    platform (enum: telegram)
  \item
    enabled (boolean)
  \item
    created\_at, updated\_at (timestamp)
  \end{itemize}
\item
  \textbf{linking\_tokens}: Tokens de magic link

  \begin{itemize}
  \tightlist
  \item
    id (UUID, PK)
  \item
    token (varchar, unique)
  \item
    user\_id (UUID, FK)
  \item
    expires\_at (timestamp)
  \item
    used (boolean)
  \end{itemize}
\end{itemize}

\subsection{Tablas de Gestión de
Usuarios}\label{tablas-de-gestiuxf3n-de-usuarios}

\begin{itemize}
\tightlist
\item
  \textbf{users}: Cuentas de usuario

  \begin{itemize}
  \tightlist
  \item
    id (UUID, PK)
  \item
    email (varchar, unique)
  \item
    password\_hash (varchar)
  \item
    created\_at, updated\_at (timestamp)
  \end{itemize}
\end{itemize}

\section{Evolución del Esquema}\label{evoluciuxf3n-del-esquema}

Migraciones de base de datos manejadas a través de GORM AutoMigrate:

\begin{itemize}
\tightlist
\item
  \textbf{Automáticas}: Actualizaciones de esquema al inicio de la
  aplicación
\item
  \textbf{Versionadas}: Archivos de migración para cambios complejos
\item
  \textbf{Transaccionales}: Rollback seguro en fallos
\end{itemize}

\section{Optimizaciones de
Rendimiento}\label{optimizaciones-de-rendimiento}

\subsection{Índices}\label{uxedndices}

\begin{itemize}
\tightlist
\item
  Claves primarias en todas las tablas
\item
  Índices de claves foráneas
\item
  Índices compuestos en columnas consultadas frecuentemente
\item
  Índices basados en tiempo para datos históricos
\end{itemize}

\subsection{Optimización de
Consultas}\label{optimizaciuxf3n-de-consultas}

\begin{itemize}
\tightlist
\item
  Operaciones JOIN eficientes
\item
  Uso apropiado de EXPLAIN ANALYZE
\item
  Pooling de conexiones con driver pgx
\item
  Sentencias preparadas para consultas repetidas
\end{itemize}

\section{Retención de Datos}\label{retenciuxf3n-de-datos}

Políticas de retención de datos configurables:

\begin{itemize}
\tightlist
\item
  \textbf{Resultados de Verificaciones}: Ventana rodante de 30 días
\item
  \textbf{Métricas}: Datos históricos de 1 año
\item
  \textbf{Notificaciones}: Registro de auditoría de 90 días
\end{itemize}

\section{Estrategia de Backup}\label{estrategia-de-backup}

\begin{itemize}
\tightlist
\item
  \textbf{Backups Automatizados}: Instantáneas diarias
\item
  \textbf{Recuperación Point-in-Time}: Archivado WAL
\item
  \textbf{Almacenamiento Offsite}: Almacenamiento de backup en la nube
\item
  \textbf{Pruebas}: Pruebas regulares de restauración
\end{itemize}




\end{document}
